\chapter{Verortung}

Sie haben jetzt das erste Semester im Fach Phonetik und Sprachverarbeitung hinter sich. Ein Ziel dieser Lehrveranstaltung war, Ihnen einen ersten Einblick in die Disziplinen zu geben, die Sie im Laufe des Studiums vertiefen werden: \emph{Phonetik} (einschließlich der \emph{Phonologie}), \emph{Sprachtechnologie} und \emph{Psycholinguistik}. Zum Abschluss der Lehrveranstaltung werden wir noch einmal versuchen, Inhalte zu sortieren.


\newpage

\section{Übungen}


Erinnern Sie sich an die wissenschaftliche Fragestellung, die sie im Kapitel \emph{Fragestellungen und Recherche} formuliert haben. Diskutieren Sie, wie Sie diese Fragestellung heute präzise formulieren würden und notieren Sie dies:

\vspace{5cm}

Inwiefern haben Sie Ihre Erwartungen an das Studium der \emph{Phonetik und Sprachverarbeitung} verändert?

\vspace{5cm}


\section{Lesetipps für die vorlesungsfreie Zeit}

Geschrieben von einer Ikone der Experimentalphonetik, mit unterhaltsamen Ausflügen:

Ladefoged, P. (2003). \emph{Phonetic Data Analysis. An Introduction to Fieldwork and Instrumental Techniques}. Malden, MA u. a.: Blackwell.
