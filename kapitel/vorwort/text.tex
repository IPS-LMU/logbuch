\chapter*{Vorwort}

\textipa {["hE\textturna{}tslI\c{c}vIl"kOm@n]} am Institut für Phonetik und Sprachverarbeitung (IPS) an der LMU München!
Sie halten hier ein für Sie frisch gedrucktes Logbuch in Händen, und wir hoffen, dass es Sie darin unterstützt, einen Überblick über das Fach \emph{Phonetik und Sprachverarbeitung} zu bekommen.

Das Logbuch wurde im Rahmen des Multiplikatoren-Projektes erstellt, einem Teil des Qualitätspakts \emph{Lehre@LMU} der Ludwig-Maximilians-Universität München. Unter dem Motto „Für Lehre begeistern. Kompetent lehren.“ wird mit einem interdisziplinären Projekt versucht, die wissenschaftliche Vielfalt im Fach Phonetik und Sprachverarbeitung für Studierende deutlicher zu machen. An dieser Stelle gilt unser herzlicher Dank unseren engagierten studentischen Hilfskräften Katharina Juhl, Daniela Pilz und Korbinian Slavik, die durch rege Diskussion den Aufbau mitgestaltet und schlussendlich die gut lesbare Form ermöglicht (ge\TeX{}t ;-)) haben.

Das Logbuch ermöglicht Ihnen zweierlei: Zum einen erhalten Sie einen Überblick über das Fach Phonetik und Sprachverarbeitung – zum anderen ermöglicht Ihnen das Logbuch, eine wissenschaftliche Arbeitstechnik kennenzulernen. Das wissenschaftliche Logbuch dient (gleichwohl dem der Seefahrt) zur Positionsbestimmung, d.\,h. Sie bestimmen kontinuierlich Ihre wissenschaftlichen Positionen, dokumentieren damit Ihre wissenschaftliche Entwicklung. Zudem erleichtert es das wissenschaftliche Schreiben, indem Sie üben aufzuschreiben, worüber Sie nachdenken, was Sie mit KommilitonInnen diskutiert haben oder was Sie allgemein für wichtig halten.

Da Sie das Modul \textit{P1 Grundlagen} sowohl in die Phonetik als \textit{Disziplin} als auch in das \textit{Fach} Phonetik und Sprachverarbeitung einführen soll, ist der Titel dieser Lehrveranstaltung \emph{Deskriptive Phonetik} etwas weiter zu fassen. Zur Erläuterung hierzu eine kleine Anmerkung: Das IPS wurde 1972 gegründet und bestand zunächst aus den Fächern Phonetik und Sprechwissenschaft (Psycholinguistik). 1995 wurde das Bayerische Archiv für Sprachsignale (BAS) als öffentliche Einrichtung der LMU gegründet und befindet sich seitdem am IPS. Die Mitarbeiter des BAS forschen v.\,a. im sprachtechnologischen Bereich. Historisch gewachsen finden sich damit an unserem Institut die Disziplinen Phonetik, Psycholinguistik und Sprachtechnologie.  Die drei Disziplinen und Teile ihrer Inhalte wurden im Zuge des Bologna-Prozesses 2009 in Module des Faches Phonetik und Sprachverarbeitung umgesetzt. Daher werden Sie in diesem Seminar nicht ausschließlich mit Inhalten der Deskriptiven Phonetik, sondern in kompakter Form mit einem Gesamtüberblick über das Fach konfrontiert. Unser Institut ist in Deutschland das einzige, an dem man Phonetik im Hauptfach studieren kann, und es zählt zu den weltweit innovativsten Forschungslabors. Es bietet Ihnen damit eine einmalig breite und gründliche Ausbildung – nutzen Sie diese Chance!

Diese Lehrveranstaltung startet zu Beginn des Semesters als Seminar, in dem Inhalte der Vorlesung aus \emph{P1} geübt und vertieft werden. Später im Semester, wenn die Kapitel zur Sprachtechnologie und Psycholinguistik erreicht werden, wird die inhaltliche Grundlage für die Übungen in einem vorlesungsähnlichen Rahmen in der Veranstaltung selbst gelegt.

In allen Kapiteln des Logbuchs finden Sie zu Beginn Cartoons, die helfen sollen, die Inhalte grafisch zu fassen und in Beziehung zu setzen. Wir sind noch auf der Suche nach einer einheitlichen \textit{Topographie}, die alle Inhalte und deren Zusammenhänge grafisch verdeutlicht. Hier sind wir noch im Prozess und freuen uns über Rückmeldungen, wie Sie für sich eine Topographie, eine grafische Verortung des Faches, vornehmen würden. Insofern verstehen Sie dieses Logbuch tatsächlich als Arbeitsbuch. Ihre Anregungen, Kommentare, Korrekturen sind uns gerne willkommen.

So, nun sind wir gespannt, ob dieses Logbuch einen Beitrag dazu leisten kann, unserem Ziel näher zu kommen, Ihnen eine Einführung und einen Überblick über das Fach Phonetik und Sprachverarbeitung zu ermöglichen und bereits erste Zusammenhänge zwischen den einzelnen Disziplinen aufzudecken.

Einen guten Start für Ihre wissenschaftliche Reise!

\begin {flushright} 
Oktober 2018
\end{flushright}
Christoph Draxler, Felicitas Kleber, Florian Schiel, Anke Werani, Raphael Winkelmann, Nikola Anna Eger, Markus Jochim
