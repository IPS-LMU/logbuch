\chapter{Beschreibung von Sprachlauten in der Literatur}

Mithilfe von segmentierten und etikettierten Sprachaufnahmen ist es möglich die akustische Ausprägung von Konsonanten und Vokalen genau zu beschreiben und akustische Reize (\emph{acoustic cues}) herauszufiltern, die für die Sprachperzeption relevant sind. Die Beschreibung von Sprachlauten basiert aber in erster Linie auf artikulatorischen und perzeptiven Aspekten. Zum Glück gibt es bereits sehr gute Beschreibungen der Sprachlaute, die in den Sprachen der Welt vorkommen, sodass Sie für diese Übung nicht auf eigene Sprachaufnahmen zurück greifen müssen, sondern sich mithilfe von Fachliteratur ein Bild machen können.

\section{Stichworte zur Vorlesung \em{Konsonanten und Vokale}}
Artikulationsart und -ort, Stimmhaftigkeit, Zungenhöhe und –position, Lippenrundung, Kardinalvokale\dots $\rightarrow$ {\tt L5\underline{\ }Konsonanten{\&}Vokale.pdf}

\section{Übungen}
Lesen Sie den Ihrer Gruppe ausgeteilten Text durch und diskutieren Sie im Anschluss in der Gruppe die wichtigsten Informationen. Gehen Sie dabei auch auf die Beschreibungsebenen ein (artikulatorisch, akustisch oder perzeptiv). Im Anschluss werden die Textzusammenfassungen den anderen Gruppen in einer Kurzpräsentation vorgestellt.
\newline
