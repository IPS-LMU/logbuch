\documentclass[11pt]{book}
\usepackage{graphicx}
\usepackage{amssymb}
\usepackage[utf8]{inputenc}
\DeclareUnicodeCharacter{00A0}{ }	%Gesperrte Leerzeichen aus Word werden erkannt
\usepackage[T3,T1]{fontenc}
\usepackage[ngerman]{babel}
\usepackage[safe]{tipa}         % Phonetische Umschrift
\usepackage{pdfpages}
\usepackage{subfigure}
\usepackage{amsmath}
\usepackage{graphics}
\usepackage[all,2cell]{xy}
\usepackage[]{todonotes}
\usepackage[hidelinks]{hyperref}
\usepackage{setspace}
\usepackage{rotating}
\usepackage{tabularx}
%\usepackage{fancyhdr} Kann bei Bedarf verwendet werden. \pagestyle{plain} dann evtl. auskommentieren

\setlength{\textwidth}{15cm}
\setlength{\textheight}{22cm}
\setlength{\oddsidemargin}{0.5cm}
\setlength{\evensidemargin}{0.5cm}
\setlength{\parindent}{0cm}
\setlength{\parskip}{1.5ex plus 0.5ex minus 0.5ex}

% Abstände
%\renewcommand*\chapterheadstartvskip{\vspace*{5\baselineskip}} Funktioniert noch nicht
\setlength{\parsep}{5cm}
\setlength{\itemsep}{2cm}
\newcommand{\vsph}{\vspace*{0.5cm}}
\newcommand{\vspe}{\vspace*{1cm}}
\newcommand{\vsmh}{\vspace*{-0.5cm}}
\newcommand{\vsme}{\vspace*{-1cm}}
\newcommand{\vsmz}{\vspace*{-2cm}}
\newcommand{\vsmd}{\vspace*{-3cm}}
\newcommand{\vsmv}{\vspace*{-4cm}}

% Aufzählung
\newcommand{\begi}{\begin{itemize} \setlength{\itemsep}{0.2cm}}
\newcommand{\majit}[1]{\item \textbf{#1}}
\newcommand{\ndit}{\end{itemize}}
\newcommand{\begn}{\begin{enumerate} \setlength{\itemsep}{0.2cm}}
\newcommand{\ndn}{\end{enumerate}}
\newcommand{\itemsp}[1]{\item #1 \vspace{0.5cm}}

%Tabellen
\usepackage{rotating}
\renewcommand{\arraystretch}{2.5}

% Logo
%\newcommand{\mylg}[1]{\MyLogo{#1}}

% Farblich hervorgehobene Schrift (grü>n, rot)
\newcommand{\bfgc}[1]{\textcolor[rgb]{0.12,0.66,0}{\textbf{#1}}}
\newcommand{\bfrc}[1]{\textcolor[rgb]{1,0,0}{\textbf{#1}}}

\sloppy %Schaltet auf eine großzügige Formatierungsweise um, die relativ wenige Worttrennungen am Zeilenende erzeugt, dafür aber auch etwas größere Wortabstände innerhalb der Zeilen zuläßt.
\parskip = 0.2in
\parindent = 0.0in


\pagestyle{plain}	%Seitenzahl unten, zentriert






\begin{document}

\begin{titlepage}
\begin{center}
% Oberer Teil der Titelseite:
\begin {figure}[t]
\includegraphics[width=1.0\textwidth] {grafiken/titel/landkarte} \\
\end{figure} 

% Title
\newcommand{\HRule}{\rule{\linewidth}{0.5mm}}
\HRule \\[0.3cm]
{ \huge \bfseries Logbuch Phonetik und Sprachverarbeitung}\\
{\bfseries P1.3 Deskriptive Phonetik 2018/19}
\\
\HRule \\[0.5cm]

% Author and supervisor
\begin{minipage}{0.8\textwidth}
%\begin{flushleft} 
\large
%Autoren:\\
Felicitas Kleber, Florian Schiel, Raphael Winkelmann, Christoph Draxler, Anke Werani, Nikola Anna Eger, Markus Jochim

%\end{flushleft}
\end{minipage}
%\fill

% Unterer Teil der Seite
{\large \today}\vspace {1cm}

\begin{minipage}[cb]{0.1\textwidth}
\includegraphics[width=\textwidth]{grafiken/titel/bmbf.jpg}
\end{minipage}
\begin{minipage}[cb]{0.1\textwidth}
\includegraphics[width=\textwidth]{grafiken/titel/center.jpg}
\end{minipage}
\begin{minipage}[cb]{0.1\textwidth}
\includegraphics[width=\textwidth]{grafiken/titel/lehreatlmu.jpg}
\end{minipage}
\begin{minipage}[cb] {0.1\textwidth}
\includegraphics[width=\textwidth]{grafiken/titel/ips.png}
\end{minipage}


\begin{spacing}{0.2}
\tiny Dieses Teilprojekt wird im Rahmen des Multiplikatoren-Projekts durch den Qualitätspakt Lehre (Lehre@LMU) gefördert. Das Multiplikatoren-Projekt ist an das LMU Center for Leadership and People Management angegliedert. Der Qualitätspakt Lehre wird aus Mitteln des Bundesministeriums für Bildung und Forschung unter dem Förderkennzeichen 01PL12016 gefördert. Die Verantwortung für den Inhalt dieser Veröffentlichung liegt beim Autor \normalsize
\end{spacing}


\newpage

\textbf {Nicht vergessen:\\Erster Stammtisch des Semesters am\\Donnerstag, den 25.10.2018 um 18 Uhr in der Phonetik-Bibliothek!}
%\vspace*{3cm}
%\includegraphics [width=\textwidth]{Abb/Notizzettel.jpg}

\end{center}
\end{titlepage}
\thispagestyle{empty}
\tableofcontents

\chapter*{Vorwort}

\textipa {["hE\textturna{}tslI\c{c}vIl"kOm@n]} am Institut für Phonetik und Sprachverarbeitung (IPS) an der LMU München!
Sie halten hier ein für Sie frisch gedrucktes Logbuch in Händen, und wir hoffen, dass es Sie darin unterstützt, einen Überblick über das Fach \emph{Phonetik und Sprachverarbeitung} zu bekommen.

Das Logbuch wurde im Rahmen des Multiplikatoren-Projektes erstellt, einem Teil des Qualitätspakts \emph{Lehre@LMU} der Ludwig-Maximilians-Universität München. Unter dem Motto „Für Lehre begeistern. Kompetent lehren.“ wird mit einem interdisziplinären Projekt versucht, die wissenschaftliche Vielfalt im Fach Phonetik und Sprachverarbeitung für Studierende deutlicher zu machen. An dieser Stelle gilt unser herzlicher Dank unseren engagierten studentischen Hilfskräften Katharina Juhl, Daniela Pilz und Korbinian Slavik, die durch rege Diskussion den Aufbau mitgestaltet und schlussendlich die gut lesbare Form ermöglicht (ge\TeX{}t ;-)) haben.

Das Logbuch ermöglicht Ihnen zweierlei: Zum einen erhalten Sie einen Überblick über das Fach Phonetik und Sprachverarbeitung – zum anderen ermöglicht Ihnen das Logbuch, eine wissenschaftliche Arbeitstechnik kennenzulernen. Das wissenschaftliche Logbuch dient (gleichwohl dem der Seefahrt) zur Positionsbestimmung, d.\,h. Sie bestimmen kontinuierlich Ihre wissenschaftlichen Positionen, dokumentieren damit Ihre wissenschaftliche Entwicklung. Zudem erleichtert es das wissenschaftliche Schreiben, indem Sie üben aufzuschreiben, worüber Sie nachdenken, was Sie mit KommilitonInnen diskutiert haben oder was Sie allgemein für wichtig halten.

Da Sie das Modul \textit{P1 Grundlagen} sowohl in die Phonetik als \textit{Disziplin} als auch in das \textit{Fach} Phonetik und Sprachverarbeitung einführen soll, ist der Titel dieser Lehrveranstaltung \emph{Deskriptive Phonetik} etwas weiter zu fassen. Zur Erläuterung hierzu eine kleine Anmerkung: Das IPS wurde 1972 gegründet und bestand zunächst aus den Fächern Phonetik und Sprechwissenschaft (Psycholinguistik). 1995 wurde das Bayerische Archiv für Sprachsignale (BAS) als öffentliche Einrichtung der LMU gegründet und befindet sich seitdem am IPS. Die Mitarbeiter des BAS forschen v.\,a. im sprachtechnologischen Bereich. Historisch gewachsen finden sich damit an unserem Institut die Disziplinen Phonetik, Psycholinguistik und Sprachtechnologie.  Die drei Disziplinen und Teile ihrer Inhalte wurden im Zuge des Bologna-Prozesses 2009 in Module des Faches Phonetik und Sprachverarbeitung umgesetzt. Daher werden Sie in diesem Seminar nicht ausschließlich mit Inhalten der Deskriptiven Phonetik, sondern in kompakter Form mit einem Gesamtüberblick über das Fach konfrontiert. Unser Institut ist in Deutschland das einzige, an dem man Phonetik im Hauptfach studieren kann, und es zählt zu den weltweit innovativsten Forschungslabors. Es bietet Ihnen damit eine einmalig breite und gründliche Ausbildung – nutzen Sie diese Chance!

Diese Lehrveranstaltung startet zu Beginn des Semesters als Seminar, in dem Inhalte der Vorlesung aus \emph{P1} geübt und vertieft werden. Später im Semester, wenn die Kapitel zur Sprachtechnologie und Psycholinguistik erreicht werden, wird die inhaltliche Grundlage für die Übungen in einem vorlesungsähnlichen Rahmen in der Veranstaltung selbst gelegt.

In allen Kapiteln des Logbuchs finden Sie zu Beginn Cartoons, die helfen sollen, die Inhalte grafisch zu fassen und in Beziehung zu setzen. Wir sind noch auf der Suche nach einer einheitlichen \textit{Topographie}, die alle Inhalte und deren Zusammenhänge grafisch verdeutlicht. Hier sind wir noch im Prozess und freuen uns über Rückmeldungen, wie Sie für sich eine Topographie, eine grafische Verortung des Faches, vornehmen würden. Insofern verstehen Sie dieses Logbuch tatsächlich als Arbeitsbuch. Ihre Anregungen, Kommentare, Korrekturen sind uns gerne willkommen.

So, nun sind wir gespannt, ob dieses Logbuch einen Beitrag dazu leisten kann, unserem Ziel näher zu kommen, Ihnen eine Einführung und einen Überblick über das Fach Phonetik und Sprachverarbeitung zu ermöglichen und bereits erste Zusammenhänge zwischen den einzelnen Disziplinen aufzudecken.

Einen guten Start für Ihre wissenschaftliche Reise!

\begin {flushright} 
Oktober 2018
\end{flushright}
Christoph Draxler, Felicitas Kleber, Florian Schiel, Anke Werani, Raphael Winkelmann, Nikola Anna Eger, Markus Jochim





\chapter{Allgemeine Einführung}

Im weitesten Sinne kann man die Phonetik als die Wissenschaft der gesprochenen Sprache beschreiben. In der ersten Sitzung der Grundlagen-Vorlesung „Phonetik“ wurde eine Reihe differenzierter Definitionen präsentiert, bei denen vor allem die Sprachproduktion (\textit{Artikulation}), Transmission (\textit{Akustik}) und \textit{Perzeption} im Vordergrund stehen. Bei der Erforschung gesprochener Sprache kommen \textit{signalphonetische} und \textit{symbolphonetische} Untersuchungsansätze zur Anwendung. $\rightarrow$ {\tt L1\underline{\ }Einführung.pdf} 

Die \textit{Sprachkette} und das \textit{signalphonetische} Band (Abb.~\ref{fig1}) stellen die oben genannten Bereiche der Phonetik graphisch dar. Des Weiteren können in ihr die Neurophonetik, die sich mit der Sprechplanung und Sprachverarbeitung im Gehirn beschäftigt, sowie die Psycholinguistik, die unter anderem die menschliche Sprachfähigkeit untersucht, verortet werden.  Um den Bereich der Sprachtechnologie in der Sprachkette abzubilden müssen entweder der Sprecher (Sprachsynthese) und/oder der Hörer durch Computer   ausgetauscht werden. Das Prinzip der Sprachkette bleibt bestehen, aber die Komponenten müssen geändert bzw. ergänzt werden und Untersuchungsgegenstände ändern sich teilweise.

Während die P1-Vorlesung vorrangig Artikulation, Akustik und Perzeption und darüber hinaus die Phonologie, d.\thinspace h. die Lehre von den Funktionen von Lauten in einer Sprache, behandelt, werden in dieser Übung auch die anderen oben genannten Aspekte der Sprachkette beleuchtet und in einen phonetischen Gesamtrahmen eingebettet.

\begin{figure}[htbp]
\begin{center}
\includegraphics[width=\textwidth]{grafiken/allgemeine-einfuehrung/speech-chain}
\caption{nach Denes, P. B.; Pinson, E. N. (2012). The Speech Chain. The Physics And Biology Of Spoken Language.}
\label{fig1}
\end{center}
\end{figure}

\section{Stichworte zur ersten Stunde der Vorlesung} 

ABC-Prosodie,  Symbol- vs. Signalphonetik, Signalphonetisches Band, Sprachkette, Sprachlaut... $\rightarrow$ {\tt L1\underline{\ }Einführung.pdf} 


\section{Literatur}

Borden, G. J.; Harris, K. S., Raphael, L. J. (2011). \emph{Speech science primer. Physiology, acoustics, and perception of speech.} 6. Aufl. Baltimore: Lippincott Williams \& Wilkins. \newline\\
Bußmann, H. (2008). \emph{Lexikon der Sprachwissenschaft.} 4. Aufl. Stuttgart: Kröner. \newline\\
Clark, J. \& Yallop, C. (2007). \emph{An introduction to phonetics and phonology.} Malden u.\,a.:  Basil Blackwell.\newline\\
Gussenhoven, C. \& Jacobs, H. (2011). \emph{Understanding Phonology.} London  u.\,a.: Hodder.\newline\\
Hall, T. A. (2011).\emph{ Phonologie: Eine Einführung.} 2. Aufl. Berlin: Walter de Gruyter. \newline\\
International Phonetic Association (Hrsg.)(1999). \emph{Handbook of the International Phonetic Association}. Cambridge: Cambridge University Press. \newline\\
Kohler, K. J.(1995). \emph{Einführung in die Phonetik des Deutschen.} Berlin: Erich Schmidt. \newline\\
Ladefoged, P. (2003). \emph{Phonetic Data Analysis. An Introduction to Fieldwork and Instrumental Techniques}. Malden, MA u. a.: Blackwell.\newline\\
Ladefoged, P.; Johnson, K.  (2014). \emph{A course in phonetics.} 7.\,Aufl. Stamford u.\,a.: Cengage Learning. \newline\\
Ladefoged, P. \& Ferrari, S. (2012). \emph{Vowels and consonants.} An introduction to the sounds of language. 3.\,Aufl. Chichester: Wiley-Blackwell. \newline\\
Pompino-Marschall, B. (2009). \emph{Einführung in die Phonetik.} 3. Aufl. Berlin/New York: de Gruyter.  \newline\\
Reetz, H. (2003). \emph{Artikulatorische und akustische Phonetik.} 2.\,Aufl. Trier: Wissenschaftlicher Verlag.  \newline\\
Spencer, A. (1996). \emph{Phonology.} Oxford: Blackwell.\newline\\
Tillmann, H. G. \& Mansell, P. (1980). \emph{Phonetik: Lautsprachliche Zeichen, Sprachsignale und lautsprachlicher Kommunikationsprozeß.} Stuttgart: Klett-Cotta. \newline\\

\section{Interaktives Glossar}

Im Laufe des Semesters werden Ihnen einige Begriffe begegnen, über die Sie mehr wissen wollen oder mit denen Sie noch nicht soviel anfangen können. Hier haben Sie die Möglichkeit sich diese Wörter zu notieren und selbstständig, bzw. mit unserer Hilfe, Erklärungen zu finden. Glossare, Indexe oder Register finden sich in vielen der oben genannten Bücher; z.\,B. in Pompino-Marschall\,(2009), Reetz\,(2003), Draxler\,(2008) oder in Ladefoged \& Ferrari\,(2012). Bei Fragen stehen wir ihnen gerne zur Verfügung.  





\chapter{Sprechen}

\section{Stichworte zur Vorlesung \em{Artikulatorische Phonetik}} 

Sprechen als aerodynamischer Prozess, Kehlkopf/Larynx, Stimmlippen(schwingung), Phonation, Vokaltrakt... $\rightarrow$ {\tt L2\underline{\ }Artikulation.pdf}

\begin{figure}[htbp]
\begin{center}
\includegraphics[width=\textwidth]{grafiken/sprechen/sprechen}
\label{t1}
\end{center}
\end{figure}



\section{Übungen}

1.	Wie viele Sprachlaute gibt es ihrer Meinung nach im Deutschen? Listen Sie sie auf und notieren Sie Beispielwörter, in denen diese vorkommen.
\vspace*{7cm}


2. Beschriften Sie bitte die nachfolgende Abbildung.
\begin{figure}[htbp]
\begin{center}
\includegraphics[scale=3]{grafiken/sprechen/kehlkopf}
\caption{Kehlkopf}
\label{fig2}
\end{center}
\end{figure}

\newpage
3.	Benennen Sie die nummerierten Bereiche des nachfolgend abgebildeten Vokaltraktes:
\begin{figure}[htbp]
\begin{center}
\includegraphics[scale=0.6]{grafiken/sprechen/kopf}
\caption{©  Tavin /Wikimedia Commons / CC-BY-3.0}
\label{fig3}
\end{center}
\end{figure}

4.	Welche Sprechorgane neben den Stimmlippen haben aerodynamische Funktionen?
\begin{figure}[htbp]
\begin{center}
\includegraphics[width=\textwidth]{grafiken/sprechen/ipa}
\caption{IPA-Tabelle}
\label{fig4}
\end{center}
\end{figure}








\chapter{Fragestellungen und Recherche}


\section{Übungen}

1.	Die Gründe für eine Studienfachwahl und die Erwartungen an das letztendlich gewählte Fach sind vielfältig.  Welche Fragen stellen sich Ihnen, wenn Sie an Phonetik denken? Worauf hätten Sie gerne eine Antwort, die man ihrer Meinung nach nur mit phonetischem Fachwissen geben kann? Diskutieren Sie in der Gruppe Ihre Fragen an die Phonetik und wählen Sie gemeinsam eine aus, die Sie besonders interessant finden.\\
\newpage
a.	Formulieren Sie die Fragestellung auf eine präzise Weise.

\vspace*{5cm}


b.	Suchen Sie mithilfe Ihrer elektronischen Geräte im Online Katalog der Universitätsbibliothek nach Büchern, die Ihnen bei der Klärung der Frage helfen können. Notieren Sie Stichworte und Suchergebnisse.
\vspace*{5cm}



\chapter{Sprachaufnahme: Sprechen erfassen und sichtbar machen} 

Gesprochene Sprache ist flüchtig. Um die Transmission von Sprachschall zwischen Schallquelle (z.\thinspace B. Sprecher*in) und Schallsenke (z.\thinspace B. Hörer*in) genau untersuchen zu können (also nicht mittels synchroner Transkription), muss der Sprachschall durch eine Sprachaufnahme festgehalten werden (vgl. auch Kapitel Sprachdatenbanken). 

\section{Stichworte zur Vorlesung \em{Sprachakustik}}

Schall, Schwingung, akustische Signaltypen, (Grund-)Frequenz, Formanten, Oszillogramm, Spektrum, Spektrogramm, Quelle-Filter... $\rightarrow$ {\tt L3\underline{\ }Akustik.pdf}
\begin{figure}[htbp]
\begin{center}
\includegraphics[width=\textwidth]{grafiken/sprachaufnahme/sprechen-erfassen}
\label{t2}
\end{center}
\end{figure}

\section{Übungen}

\subsection*{Aufnahmeübung}


Die Sprachaufnahme erfolgt in Kleingruppen an den mitgebrachten Laptops. Jede Gruppe erhält einen schriftlichen Arbeitsauftrag und ggf. ein Mikrofon oder Aufnahmegerät. Führen Sie die Aufnahme wie angegeben durch und sichern Sie die Audiodateien. In der nächsten Stunde werden wir auf die Dateien zurückgreifen.
\vspace*{7cm} 

Tipp: \newline \\ 

\begin {minipage} {0.1\textwidth}
\includegraphics[width=\textwidth]{grafiken/sprachaufnahme/praat.png}
\end{minipage}
\hspace {1cm}
\begin{minipage} {0.7\textwidth}
Praat können Sie kostenlos unter {\tt www.fon.hum.uva.nl/praat/} (Stand: 12.10.2018) herunterladen. 
\end {minipage}


\begin {minipage} {0.1\textwidth}
\includegraphics[width=\textwidth]{grafiken/sprachaufnahme/speechrecorder.png}
\end{minipage}
\hspace{1cm}
\begin{minipage} {0.6\textwidth}
Ebenfalls kostenlos herunterladen können Sie verschiedene Aufnahmeprogramme, die sich für phonetische Zwecke eignen, darunter SpeechRecorder ({\tt www.speechrecorder.org}), Audacity  ({\tt www.audacityteam.org}) und Praat (siehe oben) (Links Stand: 12.10.2018).
\end {minipage}



\begin{figure}[htb]
\begin{center}
\includegraphics[width=1.2\textwidth]{grafiken/sprechen/kreuzwortraetsel}
\caption{Kreuzworträtsel}
\label{fig5}
\end{center}
\end{figure}


\chapter{Annotationstypen}

Sprachaufnahmen können sowohl symbol- als auch signalphonetisch analysiert werden. In der Regel, insbesondere im Vorfeld einer Datenbankerstellung, werden die Sprachaufnahmen anschließend segmentiert, also in kleinere sprachliche Einheiten (z.\thinspace B. Wörter, Silben, Laute etc.) zerlegt und etikettiert, d.\thinspace h. mit Symbolen versehen. Für die Etikettierung stehen verschiedene Annotationstypen zur Verfügung. Die Segmentierung und Etikettierung basiert häufig auf unserer auditiven Perzeption.

\section{Stichworte zur Vorlesung \em{Sprachperzeption}}

Gehörknöchelchen, Hebelwirkung, Cochlea, Haarzellen, Wanderwelle, Psychoakustik, Kategoriale Wahrnehmung\dots $\rightarrow$ {\tt L4\underline{\ }Perzeption.pdf}

\begin{figure}[htbp]
\begin{center}
\includegraphics[width=\textwidth]{grafiken/annotationstypen/annotationstypen}
\label{t3}
\end{center}
\end{figure}
\section{Übungen}
 
1.	Öffnen Sie den ersten Satz aus ihrem Dialog in Praat ($\rightarrow$ {\tt praat.pdf}) und erzeugen Sie ein TextGrid-File mit zwei \emph{interval tiers} (1 = Wort, 2 = Segment). Öffnen Sie beide Dateien und segmentieren Sie auf der oberen Ebene (\emph{Wort-tier}) die Wörter und auf der unteren Ebene (\emph{Segment-tier}) die Sprachlaute. Für die Etikettierung der Wörter und Laute verwenden Sie bitte die Buchstaben des Deutschen. Notieren Sie ihre Beobachtung zu folgenden Fragen.\\
\newline

a.	Wie genau lassen sich Grenzen zwischen zwei Wörtern bzw. zwischen zwei Sprachlauten setzen? Notieren Sie einfache und schwierige Beispiele. \vspace{5cm}\\
b.	Wie stark hilft Ihnen ihre Sprachperzeption beim Segmentieren? \vspace{5cm}\\
\newpage
c.	Ist die Etikettierung der Sprachlaute immer eindeutig? \vspace{5cm}\\
d.	Wie gut eignen sich die Buchstaben für die Etikettierung? \vspace{5cm}\\
e.	Welche anderen akustischen Eigenschaften in der Sprachaufnahme könnte man etikettieren? \vspace*{5cm}\\






\chapter{Beschreibung von Sprachlauten in der Literatur}

Mithilfe von segmentierten und etikettierten Sprachaufnahmen ist es möglich die akustische Ausprägung von Konsonanten und Vokalen genau zu beschreiben und akustische Reize (\emph{acoustic cues}) herauszufiltern, die für die Sprachperzeption relevant sind. Die Beschreibung von Sprachlauten basiert aber in erster Linie auf artikulatorischen und perzeptiven Aspekten. Zum Glück gibt es bereits sehr gute Beschreibungen der Sprachlaute, die in den Sprachen der Welt vorkommen, sodass Sie für diese Übung nicht auf eigene Sprachaufnahmen zurück greifen müssen, sondern sich mithilfe von Fachliteratur ein Bild machen können.

\section{Stichworte zur Vorlesung \em{Konsonanten und Vokale}}
Artikulationsart und -ort, Stimmhaftigkeit, Zungenhöhe und –position, Lippenrundung, Kardinalvokale\dots $\rightarrow$ {\tt L5\underline{\ }Konsonanten{\&}Vokale.pdf}

\section{Übungen}
Lesen Sie den Ihrer Gruppe ausgeteilten Text durch und diskutieren Sie im Anschluss in der Gruppe die wichtigsten Informationen. Gehen Sie dabei auch auf die Beschreibungsebenen ein (artikulatorisch, akustisch oder perzeptiv). Im Anschluss werden die Textzusammenfassungen den anderen Gruppen in einer Kurzpräsentation vorgestellt.
\newline





\chapter{Stimmhaftigkeit}

\section{Stichworte zur Vorlesung \em{Aerodynamische Prozesse und Phonation}}

ingressiv, egressiv, Ejektiv, Implosiv, Klicklaut, Bernoulli-Prinzip, Voice onset time (VOT), supra- und subglottaler Luftdruck\dots $\rightarrow$ {\tt L6\underline{\ }Phonation.pdf}

\begin{figure}[htbp]
\begin{center}
\includegraphics[width=\textwidth]{grafiken/stimmhaftigkeit/stimmhaftigkeit}
\label{t4}
\end{center}
\end{figure}


\newpage
\section{Übungen}

1.	Suchen Sie in Ihren Aufnahmen nach Segmenten, die sie mit einem Symbol etikettiert haben, das Stimmhaftigkeit signalisiert (z.\,B. b, a, n). In welchen Fällen handelt es sich um phonetisch stimmhafte Laute, in welchen nicht?
\vspace{4cm}

2.	Gibt es eine Systematik zwischen stimmhaften Etiketten und fehlender phonetischer Stimmhaftigkeit?
\vspace{4cm}










\chapter{Dialog}
Gesprochene Sprache tritt besonders häufig in Form von \textit{Spontansprache} (z.\,B. im Gegensatz zu Lesesprache) auf, die zudem meistens Teil eines Dialogs ist. Sprachlaute, die in \textit{fließender Rede} in Kombination mit anderen Lauten geäußert werden, weichen in ihrer Form in der Regel von isoliert produzierten Sprachlauten ab (vgl. auch das Kapitel Spontansprachliche Vorgänge), die wiederum der phonetischen Beschreibung zugrunde liegen. Dialoge sind in vielerlei Hinsicht dafür geeignet, spontansprachliche und \textit{suprasegmentale} Eigenschaften in der sprachlichen Kommunikation zu untersuchen, z.\,B. prosodische Mittel zum Halten oder Beenden von \textit{Turns}. Auch in der Sprachtechnologie ($\rightarrow$ Spracherkennung und –synthese) werden zunehmend dialogbasierte Daten berücksichtigt. Viele Kommunikationsmodelle basieren ebenfalls auf dialogisierter Sprache ($\rightarrow$ Psycholinguistik).

\section{Stichworte zur Vorlesung \em{Prosodie und Intonation}}
Grundfrequenz, Tonhöhe, Wortbetonung, Satzakzentuierung, Phrasierung, Deklination, akzent- vs. silbenzählende Sprachen, Paralinguistik... $\rightarrow$ {\tt L7\underline{\ }Prosodie{\&}Intonation.pdf}

\begin{figure}[htbp]
\begin{center}
\includegraphics[width=0.6\textwidth]{grafiken/dialog/dialog}
\label{t5}
\end{center}
\end{figure}


\newpage
\section{Übungen}
1.	Was ist ein Turn? Recherchieren und notieren Sie eine Definition.
\vspace*{3cm}\\

2.	Suchen Sie in Ihren Aufnahmen nach Beispielen für Phrasierungen, die mit der Grammatik (syntaktische Phrase, Komma etc.) (a)  übereinstimmen und (b) von ihr abweichen.
\vspace*{3cm}\\
3.	Prüfen Sie in ihren Aufnahmen, ob Fragen mit einer steigenden oder anderen Grundfrequenz produziert werden.  Notieren Sie Beispiele und beschreiben sie den f0-Verlauf.
\vspace*{3cm}\\

4.	Bei welchen anderen Bereichen der Linguistik sehen Sie Schnittstellen mit der Prosodie und Intonation? Begründen Sie die Vorschläge.\vspace*{1cm}\\








\chapter{Abstrahieren} 

Bei der Analyse und Beobachtung artikulatorischer, akustischer und perzeptiver Aspekte gesprochener Sprache fällt die große Variabilität auf. Sprecher unterscheiden sich in ihrer Produktion (und folglich Akustik) von Sprachlauten und auch innerhalb eines Sprechers gibt es jede Menge Unterschiede: Kein Laut  wird jemals identisch produziert. Um die Funktion von Sprachlauten innerhalb einer Sprache und deren Lautinventar zu ermitteln, ist es nötig, die Variation auszuklammern und zu abstrahieren.

\section{Stichworte zur Vorlesung \em{Einführung in die Phonologie}}

Kontrast, Phon, Phonem, Allophon, freie Variante, komplementäre Distribution\dots $\rightarrow$ {\tt L8\underline{\ }Phonologie.pdf}

\begin{figure}[htbp]
\begin{center}
\includegraphics[width=0.6\textwidth]{grafiken/abstrahieren/distinktive-merkmale}
\label{t6}
\end{center}
\end{figure}




\newpage
\section{Übungen}

1.	Lesen Sie den ausgeteilten Text von Trubetzkoy und stellen sie die ihrer Meinung nach wichtigsten Informationen schematisch dar.\vspace{\fill}


2.	Tragen Sie die Sprachlaute des Deutschen in folgende Tabelle ein und unterstreichen Sie alle Laute, die ihrer Meinung nach Phoneme des Deutschen sind.

\begin{sidewaystable}\centering
\small
\begin{tabular}{|l|l|l|lcr|l|l|l|l|l|l|}  \hline
 & Bilabial & Labiodental & Dental & Alveolar & Postalveolar & Retroflex & Palatal & Velar & Uvular & Pharyngeal & Glottal\\ \hline
Plosive	&	&	&	&	&	&	&	&	&	&	& \\ \hline
Nasal	&	&	&	&	&	&	&	&	&	&	& \\ \hline
Tap or Flap	&	&	&	&	&	&	&	&	&	&	& \\ \hline
Fricative	&	&	&	&	&	&	&	&	& 	&	& \\ \hline
Lateral fricative	&	&	&	&	&	&	&	&	&	&	& \\ \hline
Approximant	&	&	&	&	&	&	&	&	&	&	& \\ \hline
Lateral approximant	&	&	&	&	&	&	&	&	&	&	& \\ \hline


\end{tabular}
\end{sidewaystable}

\newpage
3.	Lesen Sie den ausgeteilten Text von Kohler. Ist der Glottalverschluss Ihrer Meinung nach ein Phonem des Deutschen? Notieren Sie Pro- und Contra-Argumente. \vspace{5cm}\\







\chapter{Spontansprachliche Vorgänge}

Auch wenn Spontansprache durch viel Variabilität charakterisiert ist, so sind viele spontansprachliche Vorgänge dennoch regelhaft. Diese lassen sich als phonologische Prozesse und Regeln beschreiben.

\section{Stichworte zur Vorlesung \em{Phonologische Prozesse und Regeln}}
Neutralisierung, Assimilation, Fortisierung/Lenisierung, Epenthese, Tilgung, Längung, Kürzung, Metathese, Reduplikation $\rightarrow$ {\tt L9\underline{\ }Spontansprache.pdf}

\begin{figure}[htbp]
\begin{center}
\includegraphics[width=\textwidth]{grafiken/spontansprachliche-vorgaenge/phonologische-prozesse}
\label{t7}
\end{center}
\end{figure}

\newpage
\section{Übungen}
1.	Identifizieren und erklären Sie die phonologischen Prozesse in den Daten:

\begin{center}
\begin{tabular} {|p{3cm}|p{3cm} |l|l|} \hline
peel [pi:l]&pool [p\super{w}u:l]\\ \hline
tea [ti:]&two [t\super{w}u:]\\ \hline
she [\textesh i:]&shoe [\textesh \super{w}u:]\\ \hline
leek [li:k]&Luke [l\super{w}u:k]\\ \hline
get [get]&got [g\super{w}ot]\\ \hline
\end{tabular}\vspace{1cm}\\
\end{center}

\begin{center}
\begin{tabular} {|p{3cm}|p{3cm} |l|l|} \hline
in-legal&illegal\\ \hline
in-licit&illicit\\ \hline
in-rational&irrational\\ \hline
in-revocable&irrevocable\\ \hline
in-possible&impossible\\ \hline
in-polite&impolite\\ \hline
in-patient&impatient\\ \hline
\end{tabular}
\end{center}



\chapter{Sprachdatenbanken}

\section{Stichworte zum Vortrag \em{Sprachdatenbanken}}

Archivierung, Annotation, Primär-, Sekundär- und Metadaten, Korpus, Repository, Validierung, Prävalidierung, Datensammlung, Datenaufbereitung

\section{Übungen}

\begin{figure}[htbp]
\begin{center}
\includegraphics[width=1\textwidth]{grafiken/sprachdatenbanken/workflow-de-2}
\caption{Arbeitsablauf und Daten bei der Erstellung und Nutzung von Sprachdatenbanken. In den Kästchen stehen die Namen von Tools, die in diesem Arbeitsschritt häufig verwendet werden.}
\label{fig_sdb_arbeitsablauf}
\end{center}
\end{figure}

\newpage

\begin{enumerate}
\item{Welche Vorteile bieten web-basierte Sprachaufnahme und -annotation gegenüber traditionellen Verfahren?}
\vspace{5cm}
\item{Warum ist im Diagramm des Arbeitsablaufs (Abb. \ref{fig_sdb_arbeitsablauf}) ein Rückwärtspfeil von Annotation zu Signalverarbeitung?}
\vspace{5cm}
\item{Sind Sekundärdaten erweiterbar oder veränderbar? Wenn ja, wieso?}
\vspace{5cm}
\end{enumerate}

\newpage

\section{Literatur}

Bird, S.; Liberman, M. (2001). A Formal Framework for Linguistic Annotation. Speech Communication Band 33 Nr. 1,2. S.\,23-60. \newline\\
Bussmann, H. (1990). Lexikon der Sprachwissenschaft. Stuttgart: Körner Verlag.\newline\\
Draxler, C. (2008). Korpusbasierte Sprachverarbeitung. Tübingen: Narr Verlag. \newline\\
Schiel, F.; Draxler, C.; Baumann, A.; Ellbogen, T.; Steffen, A. (2003). The Production of Speech Corpora. Institut für Phonetik. LMU München.\newline\\
Schiel, F. (2003). The Validation of Speech Corpora. Institut für Phonetik. LMU München.


%\section{Links}
\label{link_clarin_repository}
{\tt clarin.phonetik.uni-muenchen.de/BASRepository/} (Stand: \today) Repository des Bayerischen Archivs für Sprachsignale am Institut für Phonetik der LMU.\newline\\
{\tt www.clarin.eu/content/virtual-language-observatory} (Stand: \today) Facettensuche nach Sprachdatenbanken.\newline\\
{\tt ldc.upenn.edu} (Stand: \today) Linguistic Data Consortium an der Universität von Pennsylvania.\newline\\
{\tt catalog.elra.info} (Stand: \today) Online Katalog der European Language Resources Asscociation.\newline\\
{\tt wals.info} (Stand: \today) World Atlas of Linguistic Structures online zur Dokumentation von Sprachstrukturen.








\chapter{Automatische Spracherkennung}

\includegraphics[width=\textwidth]{grafiken/automatische-spracherkennung/block.png}


\section{Stichworte zum Vortrag \em{Automatische Spracherkennung}}

Frontend, Backend, bottom-up, top-down, Merkmalsvektor, Variabilität von Sprachsignalen, Störparameter

\section{Literatur}

        {\em Gute Einführung:}\\
%        Euler S (2006): Grundkurs Spracherkennung. Vieweg Wiesbaden.\\
        Pfister, B. \& Kaufmann, T. (2008). Sprachverarbeitung - Grundlagen und 
	Methoden der Sprachsynthese und Spracherkennung. 
	Springer-Verlag Berlin Heidelberg.

%        {\em Die Verwendung von ANN in der ASR:}\\
%	Bourlard H, Morgan N (1994): Connectionist Speech Recognition - 
%	A Hybrid 
%	Approach Kluwer Academic Publishers, Engineering and Computer 
%	Science.
%       {\em Beispiel für eine spezielle Merkmalsextraktion:}\\
%	Hermansky H, Morgan N, Bayya A, Kohn P (1991): Compensation for 
%	the Effect of the Communication Channel in Auditory-Like Analysis 
%	of Speech (RASTA-PLP), EUROSPEECH 1991, Genua, p. 1367 - 1370.

 %       {\em Grundlegender Artikel über das Viterbi-Training von HMM:}\\
%	Juang B H, Rabiner L R (1990): The segmental k-means algorithm for 
%	estimating parameters of Hidden Markov Models.
%	IEEE Transactions on Acoustics, Speech and Signal Processing, Vol. 38, 
%	No. 9, Sep 1990, S. 1639 - 1641.	

        {\em Einführung in ANN:}\\
	Lippmann, R. S. (1987). An Introduction to Computing with Neural Nets.
	IEEE. ASSP Magazine. April 1987. S.\,4 - 22.

%        {\em Einführung über Dynamic Time Warping als Mustererkennung:}\\
%	Ney H, Ortmanns S (1999): Dynamic Programming Search for 
%	Continuous Speech Recognition. in: IEEE Signal Processing Magazine, 
%	Sept 1999, pp. 64-83.

 %       {\em Gute Einführung in HMM in der ASR:}\\
%	Picone J (1990): Continuous Speech Recognition Using Hidden Markov Models.
%	IEEE ASSP Magazine, Jul 1990, S. 26 - 41.

 %       {\em Grundlegender Artikel über HMM:}\\
%	Rabiner L R, Juang B H (1996): An Introduktion to Hidden Markow Modells
%	IEEE ASSP Magazine, Jan 1986, p. 4.

%	{\em Einführendes Buch über Akustische Phonetik; Kapitel 2.4 - 2.8 sind 
%	sehr gute Erklärungen zum digitalen Signal und Spektrum:}\\
%	Reetz H (2003): Artikulatorische und akustische Phonetik, 
%	Wissenschaftlicher Verlag Trier.

        {\em Gutes Lehrbuch über alle Arten von Sprachverarbeitung:}\\
        Jurafsky D. \& Martin J. H. (2000). Speech and Language Processing. 
	Prentice Hall. Kap I.7.

%        {\em Mathematische Theorie der Spracherkennung; ziemlich schwierig; 
%	interessant (auch ohne viel Mathe-Kenntnisse): Kapitel 7:}\\
%        Levinson S (2005): Mathematical Models for Speech Technology, John Wiley \& Sons, UK.

%        {\em Wissenschaftliches Arbeiten:}\\
%        Bördlein S (2002): Das Sockenfressende Monster in der Waschmaschine. Alibri Verlag Gunnar Schedel.



\chapter{Sprachsynthese}


\begin{figure}[htbp]
\begin{center}
\includegraphics[width=\textwidth]{grafiken/sprachsynthese/synthese}
\label{ts}
\end{center}
\end{figure}

\section{Stichworte zum Vortrag \em{Sprachsynthese}}

Text-to-Speech, Concept-to-Speech, Formantsynthese, artikulatorische Synthese, konkatenative Synthese, Textnormalisierung, Graphem-Phonem-Konvertierung



\section{Literatur}


\emph{Eine gute Einführung findet sich in Kapitel 8:}\\
Jurafsky, Dan \& Martin, James H. (2000): Speech \& language processing. Pearson Education India

\emph{Kapitel 3 in:}\\
Pfister B., Kaufmann T. (2008): Sprachverarbeitung - Grundlagen und 
	Methoden der Sprachsynthese und Spracherkennung. 
	Springer-Verlag Berlin Heidelberg.


\renewcommand\refname{\vskip -1cm}
\bibliography{synthese}{}
\bibliographystyle{plain}

%%%%%%%%%%%%%%%%%%%%%%%%%%%%%%
%%%%%%%%%%%%%%%%%%%%%%%%%%%%%%








\chapter{Psycholinguistik}
\begin{figure}[htbp]
\begin{center}
\includegraphics[width=0.6\textwidth]{grafiken/psycholinguistik/psycholinguistik}
\label{t9}
\end{center}
\end{figure}

\section{Stichworte zum Vortrag \em{Psycholinguistik}}

Mentales Lexikon, Spracherwerb, Sprachverarbeitung, Eye-Tracking-Methode, Lexical Decision

\newpage

\section{Übungen}



\subsection*{Fragestellungen}

Was könnten typische Fragestellungen in der Psycholinguistik sein? Welche Frage würden Sie gern untersuchen?

\vspace{4cm}

Diskutieren Sie, welche Ereignisse beobachtet und gemessen werden können. Was bedeutet das für psycholinguistische Experimente und deren Aussagen über mentale Prozesse?

\vspace{4cm}

\subsection*{Eyetracking}

Lesen Sie den Textausschnitt zum Eyetracking. Diskutieren Sie, welche Vorteile diese Methode hat und worauf beim Erstellen eines Experiments besonders geachtet werden muss.


\vspace{4cm}

\subsection*{Erstspracherwerb – Fremdspracherwerb}

Überlegen Sie in der Gruppe, was die Merkmale des Erst- und des Fremdspracherwerbs sind. Wie unterscheiden sich die beiden voneinander?

\vspace{4cm}

Überlegen Sie sich ein Beispiel und formulieren Sie eine wissenschaftliche Frage zum Fremdspracherwerb, die Sie gerne testen würden.

\vspace{4cm}

\newpage

\section{Literatur}
Aitchinson, J. (2012\super{4}). Words in the Mind: An Introduction to the Mental Lexicon. West Sussex: John Wiley \& Sons. \newline\\
Cutler, A.; Klein, W. \& Levinson, S. C. (2005). \textit{The cornerstones of twenty-first century psycholinguistics}. In: Twenty-first century psycholinguistics. Four cornerstones. Hg. von Anne Cutler.  Mahwah, New Jersey, London: Lawrence Erlbaum. 1–20.\newline\\
Dietrich, R. (2007). Psycholinguistik. Stuttgart, Weimar: J. B. Metzler.\newline\\
Göttert, K. H. (1991). Einführung in die Rhetorik. München: Wilhelm Fink Verlag.\newline\\
Huettig, F., Rommers, J., Meyer, A. S. (2001). Using the visual world paradigm to study language processing: A review and critical evaluation, Acta Psychologica, 137, Issue 2, 151-171\newline\\
Grimm, H. \& Weinert, S. (2002). \textit{Sprachentwicklung}. In: Oerter, R.; Montada, L. (Hrsg.): Entwicklungspsychologie (S. 517-550). Weinheim u.\,a.: Beltz.\newline\\
Herrmann, C.; Fiebach, C. (2004). Gehirn \& Sprache. Frankfurt am Main.\newline\\
Knobloch, Clemens (2003). \textit{Geschichte der Psycholinguistik}. In: Gerd R.; Herrmann, T. und Deutsch, W. (Hrsg.): Psycholinguistik. Psycholinguistics. Ein internationales Handbuch. An international handbook. Berlin, New York: Walter de Gruyter. 15–33.\newline\\
Koch, P. \& Oesterreicher, W. (1994). \textit{Schriftlichkeit und Sprache}. In: Günther, H.; Ludwig, O. (Hrsg.): Schrift und Schriftlichkeit. HSK. Berlin, New York: de Gruyter.\newline\\
Levelt, W. J. M.(1989). Speaking. From Intention to Articulation. Cambridge, MA; London: MIT Press.\newline\\
Osgood, C. E., und Thomas A. S. (Hrsg) (1954). \textit{Psycholinguistics. A survey of theory and research problems. Report of the 1953 summer seminar sponsored by the Committee on Linguistics and Psychology of the Social Science Research Council}. Baltimore: Waverly Press.\newline\\
Vygotskij, L. S. (1934/2002). Denken und Sprechen. Weinheim und Basel: Beltz.\newline\\



\chapter{Verortung}

Sie haben jetzt das erste Semester im Fach Phonetik und Sprachverarbeitung hinter sich. Ein Ziel dieser Lehrveranstaltung war, Ihnen einen ersten Einblick in die Disziplinen zu geben, die Sie im Laufe des Studiums vertiefen werden: \emph{Phonetik} (einschließlich der \emph{Phonologie}), \emph{Sprachtechnologie} und \emph{Psycholinguistik}. Zum Abschluss der Lehrveranstaltung werden wir noch einmal versuchen, Inhalte zu sortieren.


\newpage

\section{Übungen}


Erinnern Sie sich an die wissenschaftliche Fragestellung, die sie im Kapitel \emph{Fragestellungen und Recherche} formuliert haben. Diskutieren Sie, wie Sie diese Fragestellung heute präzise formulieren würden und notieren Sie dies:

\vspace{5cm}

Inwiefern haben Sie Ihre Erwartungen an das Studium der \emph{Phonetik und Sprachverarbeitung} verändert?

\vspace{5cm}


\section{Lesetipps für die vorlesungsfreie Zeit}

Geschrieben von einer Ikone der Experimentalphonetik, mit unterhaltsamen Ausflügen:

Ladefoged, P. (2003). \emph{Phonetic Data Analysis. An Introduction to Fieldwork and Instrumental Techniques}. Malden, MA u. a.: Blackwell.

\end{document}
